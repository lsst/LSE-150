\addtocounter{table}{-1}
\begin{longtable}{|p{0.145\textwidth}|p{0.8\textwidth}|}\hline
\textbf{Acronym} & \textbf{Description}  \\\hline

AIV & Assembly Integration and Verification \\\hline
API & Application Programming Interface \\\hline
AT & Auxiliary Telescope \\\hline
ATCS & Auxiliary Telescope Control System \\\hline
Archive & The repository for documents required by the NSF to be kept. These include documents related to design and development, construction, integration, test, and operations of the LSST observatory system. The archive is maintained using the enterprise content management system DocuShare, which is accessible through a link on the project website www.project.lsst.org \\\hline
Archive Center & Part of the LSST Data Management System, the LSST archive center is a data center at NCSA that hosts the LSST Archive, which includes released science data and metadata, observatory and engineering data, and supporting software such as the LSST Software Stack \\\hline
C & Specific programming language (also called ANSI-C) \\\hline
CCB & Change Control Board \\\hline
CCD & Charge-Coupled Device \\\hline
CI & Continuous Integration \\\hline
CSC & Controlable SAL Component \\\hline
CTIO & Cerro Tololo Inter-American Observatory \\\hline
Camera & The LSST subsystem responsible for the 3.2-gigapixel LSST camera, which will take more than 800 panoramic images of the sky every night. SLAC leads a consortium of Department of Energy laboratories to design and build the camera sensors, optics, electronics, cryostat, filters and filter exchange mechanism, and camera control system \\\hline
Center & An entity managed by AURA that is responsible for execution of a federally funded project \\\hline
DM & Data Management \\\hline
DMCS & Data Management Control System \\\hline
Document & Any object (in any application supported by DocuShare or design archives such as PDMWorks or GIT) that supports project management or records milestones and deliverables of the LSST Project \\\hline
EFD & Engineering Facilities Database \\\hline
IP & Internet Protocol \\\hline
LCR & LSST Change Request \\\hline
LOVE & LSST Operations Visualization Environment \\\hline
LSE & LSST Systems Engineering (Document Handle) \\\hline
LSST & Large Synoptic Survey Telescope \\\hline
M1M3 & Primary/Tertiary mirror \\\hline
M2 & Secondary mirror \\\hline
MT & Main Telescope \\\hline
OCS & Observatory Control System \\\hline
RPM & RPM Package Manager \\\hline
Release & Publication of a new iteration of an existing document following approval of changes through the change control process. Upon release, the new iteration becomes the current baseline and the preferred version in the archive \\\hline
SAL & Services Access Layer \\\hline
Summit & The site on the Cerro Pachón, Chile mountaintop where the LSST observatory, support facilities, and infrastructure will be built \\\hline
TCS & Telescope Control System \\\hline
TS & Test Specification \\\hline
TSS & Telescope and Site Software \\\hline
XML & eXtensible Markup Language \\\hline
YAML & Yet Another Markup Language \\\hline
configuration & A task-specific set of configuration parameters, also called a 'config'. The config is read-only; once a task is constructed, the same configuration will be used to process all data. This makes the data processing more predictable: it does not depend on the order in which items of data are processed. This is distinct from arguments or options, which are allowed to vary from one task invocation to the next \\\hline
git & A distributed revision control system, often used for software source code. See the Git User Manual for details. Not developed by LSST DM \\\hline
monitoring & In DM QA, this refers to the process of collecting, storing, aggregating and visualizing metrics \\\hline
stack & a grouping, usually in layers (hence stack), of software packages and services to achieve a common goal. Often providing a higher level set of end user oriented services and tools \\\hline
transient & A transient source is one that has been detected on a difference image, but has not been associated with either an astronomical object or a solar system body \\\hline
\end{longtable}
