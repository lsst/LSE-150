\section{Observatory Control System}\label{sect:ocs}
The LSST Observatory Control System consist of a collection of specialized components, namely; LOVE, the ScriptQueue, 
the Watcher and Control Systems. This distributed control system is designed to efficiently and safely perform astronomical 
observations individually or through automated scheduling. In this section we describe the role each of those components play 
in enabling the LSST observatory operations. 

\subsection{The ScriptQueue component} \label{sect:scriptq}
There are a number of different ways users can interact with components in the LSST system. For instance, one could easily use the 
SAL generated API in any of the supported languages to send commands directly to a single or multiple components. It is also possible 
to use SalObj Remotes to write Python scripts (e.g. SAL Scripts) that would command different components to accomplish a specified 
task. Not to mention that LOVE itself provides a customizable interface for users to interact with components.

During commissioning and operations the LSST system will require a high degree of coordination between different crews 
(different daytime and nighttime shifts, for instance), not to mention the increasing number of available components and level of 
complexity as the system ramps up. In order to manager those issues, the LSST control system contains a specialized script 
queuing component, a.k.a. the ScriptQueue\footnote{\url{https://github.com/lsst-ts/ts_scriptqueue}}.

The ScriptQueue defines an interface for developing scripts in general, and provides BaseScript, a base class for Python SAL scripts. 
As Python programs, these scripts have access to all Python functionality, both from the native Python 3 language and through 
imported modules (including \asyncio~to manage concurrent activities or libraries from the DM stack). In particular, a SAL Script 
has access to all the system components using SalObj Remotes (\secref{sect:salobj}) . Although Python is the only language 
officially supported, scripts can be written in any SAL-supported language. As long as they follow the interface defined by the 
ScriptQueue component, it should be possible to execute them.

\subsection{The Watcher}\label{sect:watcher}
The Watcher is a component that monitors the other SAL components and output alarms in a standard way that LOVE can present 
to operators. The Watcher is designed in such a way that alarm rules are easy to write and easy to understand. The rules are likely 
to evolve rapidly during commissioning and slowly after that.

Examples of alarms published by the Watcher are;
%
\begin{itemize}
\item Dangerous weather, such as rain or high humidity.
\item A SAL component is unavailable: not enabled or heartbeat is missing.
\item Actuator malfunction, such as axis motors out of closed loop, filter changer stuck, an actuator hits a limit.
\item CCD temperatures or pressures out of range.
\end{itemize}

A typical life cycle of an alarm:
\begin{enumerate}
\item Azimuth goes out of range so the controller halts motion. The Watcher reports this as an alarm with severity=serious. LOVE 
displays it.
\item An operator acknowledges the alarm to the Watcher, but the axis is still out of range. The Watcher outputs a new version of the 
alarm that includes the information that the alarm has been acknowledged. LOVE displays the alert in a way that looks less urgent 
(e.g. is grayed out). The alarm has been acknowledged but the condition is still current.
\item An operator fixes the problem and the controller reports this. The Watcher reports the alarm one last time with severity "OK". 
LOVE removes the alarm from the display.
\end{enumerate}

A typical life cycle of a transient alarm:
\begin{enumerate}
\item The azimuth drive temporarily draws too much current; the component reports this but manages to keep the axis moving 
(presumably with temporarily degraded accuracy). The Watcher reports this as an alarm with severity=serious. LOVE displays it.
\item The drive current is within normal range again before an operator has time to acknowledge the alarm. The Watcher outputs a 
new version of the alarm that says the condition is now OK but the alarm has not yet been acknowledged (a "stale alarm"). LOVE 
still displays the alarm, but in different way to indicate that the problem is gone.
\item An operator acknowledges the alarm to the Watcher. The Watcher outputs a new version of the alarm with severity "OK" 
and acknowledged=True. LOVE removes the alarm from the display.
\end{enumerate}

\subsection{High level Control Systems} \label{sect:ocs}
Given the distributed nature of the LSST system architecture it is not immediately clear that a traditional hierarchical design, 
with centralized Control System, is necessary or even desirable. A completely flat architecture seems completely workable and 
certainly sufficient during AIV and early commissioning. For example, consider a SAL Script which commands and sequences 
the telescope subsystems to move to the next field to be observed, take an exposure, and read out that exposure. The SAL Script 
can directly control each of those subsystems and maintain control of the sequencing using \asyncio. Furthermore, the complexity of 
the SAL Script can be managed through normal modular programming techniques, in which subsystem functionality is 
implemented through Python objects imported in modules.

As the system matures and knowledge is gathered about the intricate interdependencies of the various subsystems, it is possible to 
realize  that high level components, constantly monitoring the state of the observatory, can be responsible for some autonomous actions 
to safeguard operations. It is also possible to envision that some actions involving multiple components (initially developed and 
conducted by SAL Scripts) can be incorporated to one of these components. Henceforth, the role high level Control Systems will play in 
the Observatory Control System is yet to be defined once the system has matured enough.


